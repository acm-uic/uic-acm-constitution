\documentclass[12pt,titlepage]{article}
\usepackage[margin=1in]{geometry}
\usepackage{listings}
\usepackage{draftwatermark}
\usepackage{datetime}
\usepackage{fancyhdr}

\SetWatermarkText{DRAFT}
\SetWatermarkScale{4}

\fancyhf{}
\fancyhead[C]{\today\ \currenttime}
\pagestyle{fancy}

\begin{document}

\title{Constitution of the Association for Computing Machinery}
\maketitle

\tableofcontents

\pagebreak

\section{Name}

This organization shall be called the Association for Computing Machinery\\
\\
In this document, references to this organization will be made as ACM. References to the parent organization will be made as National ACM. References to this University will be made as UIC or simply the University.

\section{Purpose}

The Chapter is organized and will be operated exclusively for educational and scientific purposes and, in furtherance thereof, specific objectives are:
\begin{enumerate}
\item To promote an increased knowledge of the science, design, development, construction, language, and applications of modern computing machinery.
\item To promote a greater interest in computing machinery and its applications.
\item To promote social and professional development by providing a means of communication between persons having interest in computing machinery.
\end{enumerate}
Additional objectives may be added. However, no addition may contradict the main purpose as stated above.

\section{Membership}

Membership in ACM is limited to UIC students, faculty, and staff. ACM conforms to the policies of the University of Illinois Board of Trustees regarding nondiscrimination on the basis of gender, race, religion, physical disability, sexual orientation, or national origin.\\
\\
A student, full or part time, at UIC who is a member in good standing of the ACM may, upon request and payment of dues, become a voting member of this Chapter.\\
\\
A majority vote of a quorum of 51\% of the ACM Council may cause a student to be considered to be in poor standing. A student may also be considered in poor standing by a side effect of a Decision (section 6) made by the ACM Council. The Faculty Advisor has the power to override a Council vote to place a student in poor standing.\\
\\
A member of ACM not in good standing may not vote on any matter, hold any position in the ACM Council, nor lead a SIG.

\subsection{Dues}

Membership dues are to be paid at the start of each fall semester by each member of the ACM. If a member pays dues then that member is not required to pay any more dues if the value of dues rises before the next payment is due nor is that member entitled to a refund of the difference should dues be lowered.\\
\\
Dues must be assessed annually by the Officers to see if any changes are need based on the financial situation of the ACM.\\
\\
A change in the value of dues owed cannot effect the semester in which the change takes place. Any change in dues does not take place until the semester following the change.\\
\\
The amount in dues owed by each member may be changed by a unanimous decision between the President, Treasurer, and the Faculty Adviser.\\
\\
Any member who has not paid all dues owed by the end of the third week of the fall semester is considered in poor standing until the dues are paid.\\
\\
Any member of the ACM who is a member of National ACM and provides proof to that effect shall be exempt from paying dues for the ACM. If at anytime said National ACM membership should expire then the member must renew national membership within one week or else become a member in poor standing until either paying dues to the ACM or renewing membership with National ACM and producing proof of this renewal.

\section{ACM Council}

The main governing body of the ACM shall be the ACM Council. The ACM Council consists of the three elected Officers, a set of appointed Secretaries, a representative from each SIG, the immediate past ACM President, and the ACM's faculty adviser.\\
\\
No single person may hold more than one position on the council.

\subsection{The Officers}

At the head of the ACM Council are the three elected officers: President, Vice President, and Treasurer. The Officers together represent the official voice and position of the ACM to the University, the members, and outside organizations.

\subsubsection{President}

The President is the lead officer and thus the head of the entire ACM. It is the job of the president to have a vision for the ACM which is based on the platform on which the President was elected and to take steps to enact this vision.\\
\\
The President has final say in all officer decisions as described in section 6.1 and has the responsibility of wisely wielding that power by staying informed of the goings on in the ACM and the changes of view of the members and enacting Decisions (section 6) accordingly.\\
\\
The President must also aid the Treasurer in creating an annual budget on the timeline as described in section 4.1.3 paragraph 2.\\
\\
The President has the power to make purchases on behalf of the ACM so long as the purchase has been approved using the Decisions procedure described in section 6.\\
\\
The President and Treasurer are responsible for submitting the required \emph{Annual Report} to National ACM.

\subsubsection{Vice President}

The Vice President is second in command of the ACM. It is the responsibility of the Vice President to aid the President in leading the ACM and enacting Decisions (section 6) in the interest of the members of the ACM.\\
\\
It shall also be the responsibility of the Vice President to maintain the official Task Pool as described by section 4.2.1 and track that Tasks are being completed.

\subsubsection{Treasurer}

The Treasurer is responsible for tracking and maintaining the budget of the ACM.\\
\\
Annually before July 1 following being elected the Treasurer with the assistance of the President and with any necessary input from the rest of the ACM Council must prepare a budget for the ACM and submit the relevant portions to the Computer Science Department.\\
\\
The Treasurer has the power to make purchases on behalf of the ACM so long as the purchase has been approved using the Decisions procedure described in section 6.\\
\\
The President and Treasurer are responsible for submitting the required \emph{Annual Report} to National ACM.

\subsubsection{Officer Elections}

There shall be an annual election meeting held within two weeks of the anniversary of the previous years election meeting. The meeting may be held earlier if their is some requirement or outside pressure from the University and a Decision is made as laid out in section 6.\\
\\
The date of the meeting should be set by the start of the semester in which the election shall take place and there must be at least one announcement about the election meeting no sooner than one month before the election meeting.\\
\\
Sign ups to run for positions should be opened to members no later than 2 weeks before the election meeting.\\
\\
In order to be a candidate in elections one must be a student at the University, a member of the ACM in good standing, a member of National ACM, and sign up in advance of the election meeting using the officially supplied sign up method.\\
\\
For an election to take place there must be a quorum of at least 25 members or 25\% of the membership whichever is smaller. If quorum is not met the meeting must be rescheduled to a date within a week that can be agreed upon by a plurality of those present. If this new meeting does not have quorum but has at least 80\% of the people whom attended the previous meeting present than the election may take place.\\
\\
The election for each position will take place during the election meeting in the following order: President, Vice President, Treasurer.\\
\\
At the election meeting all candidates who registered in advance for each position should, before voting, be given an equally allotted time frame to present their platform. Following presentations all candidates should be moved out of earshot, submitting their votes before leaving, for group discussion about the candidates. After discussion voting may commence.\\
\\
The winner of an election must receive a plurality of the votes in order to win.

\subsubsection{Officer Replacements}

If the President is no longer capable of fulfilling the duties of the office of President then the Vice President shall take over the office of President and nominate a new Vice President. This new Vice President must be approved by a majority vote of the ACM Council with a quorum of 50\% of the council.\\
\\
If the Vice President is no longer capable of fulfilling the duties of the office of Vice President then the President shall nominate a new Vice President. This new Vice President must be approved by a majority vote of the ACM Council with a quorum of 50\% of the council.\\
\\
If the Treasurer is no longer capable of fulfilling the duties of the office of Treasurer then the President shall nominate a new Treasurer. This new Treasurer must be approved by a majority vote of the ACM Council with a quorum of 50\% of the council.\\
\\
Any officer replacement can be overridden by any member who produces a petition with the signatures of at least 10 members or 10\% of the membership, whichever is smaller, for the ACM Council. The action nullifies the appointment and triggers standard election procedure for that position as described by section 4.1.4 with the election meeting taking place no sooner than one week from the date of the petition's submission and notification of the meeting being sent out within 12 hours of the meeting being scheduled.

\subsubsection{Impeachment Process}

To initiate the impeachment process a member must present the Council with a petition to remove an Officer from the Officer's position containing the signatures of at least 20\% of the members.\\
\\
An agenda item must be added to the ACM Council meeting immediately following the submission of the petition to discuss the impeachment giving members on both sides the opportunity to present their case. The voting begins after this meeting takes place.\\
\\
The vote will be a Standard Vote of the Members as described in section 5 with the following addendums:

\begin{itemize}
	\item At least 50\% of members must vote.
	\item The voting will take place over the course of 7 weekdays.
	\item The decision to impeach must receive a majority of the votes.
	\item The Officer being impeached may not be the Officer in charge of collecting votes.
\end{itemize}

\subsection{Secretaries}

Secretary positions in the ACM exist to carry out and help enforce Decisions (section 6) made by the ACM Council as well as assist in the day to day management of the organization.\\
\\
Officers may grant a Secretary position to an interested member in good standing of the organization at any time. In order for this Secretary to be appointed the individual must be willing to accept one or more of the official Tasks defined in the official Task Pool.\\
\\
All Tasks accepted by the new Secretary will become that secretaries responsibility and said Secretary will be held accountable for them being carried out and for regularly updating the ACM Council on actions taken in relation to said Tasks.\\
\\
Each Secretary position should be granted a name by the Officers that is relevant to the Tasks assigned to that Secretary. As Tasks are assigned and unassigned the name may be changed by the Officers.

\subsubsection{Task Pool}

The official Task Pool is a list of responsibilities, duties, and other such tasks that are necessary for the management of the ACM or to carry out some Decision (section 6) of the ACM.\\
\\
The Task Pool must be a publicly accessible document which is to be maintained by the Vice President.\\
\\
Tasks can be added to and removed from the list by a Decision (section 6) made by the Officers. Except in the case where a task has already been assigned to a Secretary. If a task has already been assigned to a Secretary the task, and if it is the only task assigned to that Secretary then the Secretary position itself, can only be removed by a plurality vote of a quorum of 50\% of the ACM Council.

\subsubsection{Member Vote of No Confidence}

Should a member become dissatisfied with the service of one or more Secretaries that member may present to the ACM Council a petition containing signatures from 10 members or 10\% of the members, whichever is smaller. This triggers the start of a Vote of No Confidence.\\
\\
This vote shall be conducted as a Standard Vote of the Members as described in section 5.

\subsection{SIG Leaders}

SIG Leader positions on the council exist to advise the council on policies beneficial to the represented SIG, update the Council at each Council Meeting on the activities of the SIG, and to provide a communication channel between the ACM Council and the SIGs.\\
\\
Each Official SIG may send as many representatives to the Council as it wants, but each Official SIG only gets one vote on the Council. There must be at least one representative from each Official SIG sent to be a representative on the ACM Council.\\
\\
The representative should be the Leader or Leaders in charge of the SIG that is being represented except in the case where the Leader or Leaders are already members of the council in which case the SIG must send at least one representative from among the members of the SIG who is not on the Council.

\subsection{Faculty Adviser}

The Faculty Adviser exists to create a link between the ACM and the Computer Science Department as well as to provide a year to year continuity in the organization for the purposes of advising the ACM Council.\\
\\
The Faculty Adviser must be a full time faculty member in the Computer Science Department.\\
\\
The Faculty Adviser must have an \emph{ACM Professional Membership} with National ACM.\\
\\
The Faculty Adviser is considered a member in good standing of the organization however is exempt from paying dues.\\
\\
The Faculty Adviser has the power to make purchases on behalf of the ACM so long as the purchase has been approved using the Decisions procedure described in section 6 and should assist in the monitoring of the ACM's finances and ensuring that necessary payments are made.\\
\\
The Faculty Adviser should help to ensure that the ACM is operating within University rules and operating up to University standards.\\
\\
The Faculty Adviser shall be appointed by a Decision (section 6) of the ACM Council from a group of interested candidates.

\subsubsection{Removing the Faculty Adviser}
The Faculty Adviser can only be removed by a vote of the members that is initiated by a member submitting a petition containing signatures of 20\% of the members to the ACM Council.\\
\\
An agenda item must be added to the ACM Council meeting immediately following the submission of the petition to discuss the impeachment giving members on both sides the opportunity to present their case. The voting begins after this meeting takes place.\\
\\
The vote to remove shall be a Standard Vote of the Members as described in section 5 with the following addendums:

\begin{itemize}
	\item At least 50\% of members must vote.
	\item The voting will take place over the course of 7 weekdays.
	\item The decision to impeach must receive a majority of the votes.
\end{itemize}

\subsection{Past President}

The Past President is the most recent ACM President before the current one who was not removed from office by impeachment nor by being deemed incapable to fulfill the duties of the office.\\
\\
The position of Past President exists to advise the Council and to provided some continuity and assistance with knowledge from previous years.\\
\\
The Past President is considered a member in good standing of the organization even if no longer a student at the University. If the Past President is still a student then dues must be paid the sames as for all members.

\subsection{Council Meetings}

A meeting of the ACM Council must be held at least once every two weeks.\\
\\
Any voting matters or petition presentations must be made available in the form of a meeting agenda at least 24 hours before the next meeting. Any items brought up less than 24 hours before the meeting must be added to the agenda for the following meeting. Each agenda item must be given a non-zero amount of consideration during the Meeting.\\
\\
Any Officer may call a meeting of the ACM Council with at least 72 hours notice.\\
\\
Any member may call a meeting of the Council with at least one weeks notice.\\
\\
It is an expectation that all Council members make an effort to attend each Council meeting though it is not required.\\
\\
Council meetings are open to all members of the ACM.\\
\\
All those present at the meeting must be given equal opportunity to give input on each agenda item.\\
\\
No ACM Council meeting may take place without at least one of the Officers.\\
\\
The ACM President will conduct all Council meetings. In the absence of the President the Vice President will conduct the meeting. In the absence of both the President and Vice President, the Treasurer will conduct the meeting.

\section{Standard Vote of the Members}

The Standard Vote of the Members is a predefined process which may be used when a vote of the general body of the ACM is needed. It is not required but is referenced throughout this document and may be useful.\\
\\
Only members of the ACM in good standing may vote.\\
\\
Members must be given at least three weekdays notice before the start of a vote.\\
\\
Polls must be open at least five weekdays.\\
\\
An Officer will, without looking at the ballot, collect each ballot from the members as they come in and verify that each voter is in fact a member of the ACM.\\
\\
The officer in charge must establish hours during which the Officer will be available to collect ballots each voting day.\\
\\
The polls must be open for at least two hours every voting day. One of these hours must be between 9:00 and 16:00 and the other must be between 16:00 and 18:00.\\
\\
To facilitate different schedules more than one officer may be in charge of collecting votes.\\
\\
The ballots must be stored in a secure location inside a locked box. The only person who may have a key to this box is the Faculty Adviser to the ACM.\\
\\
Each member's ballot must be placed in the box by the voter and no one else.\\
\\
At least 25 members or 25\% of members, whichever is smaller, must vote in order for the vote to count otherwise the vote is canceled as if it had never occurred.\\
\\
In order to win the vote an option must receive a plurality of the votes.

\section{Decisions}

The ACM Council has the power to enact policies or rules for the organization in the form of Decisions.\\
\\
These Decisions exist to give the Council the ability to enact legislation to help manage, sustain, and adapt the ACM in a way that is best for members.\\
\\
All actions decided on by the Council must come in the form of a Decision.\\
\\
The ACM Council has the power to pass a Decision on all matters not restricted by this document.\\
\\
Each Decision must be added to the agenda of a Council Meeting at least 24 hours before the meeting.\\
\\
Any member of the ACM may present a Decision for consideration by the ACM council.

\subsection{Enacting a Decision}

The power to enact Decisions lies primarily with the Officers of the ACM.\\
\\
To enact a Decision it must pass a majority vote of all the Officers.\\
\\
The President may veto any Decision passed by the Officers.\\
\\
The ACM Council may veto any Decision made by the Officers as well as nullify a veto made by the President. To do this a majority vote must be taken of a quorum of 50\% of the ACM Council. Decisions made in this way cannot be overridden by another Decision of the Officers; it can only be overridden by a vote of the Council as described above or by a vote of the Members as described below.\\
\\
The members of the ACM may veto or nullify a veto of the ACM Council or the Officers. To do this a member must present the ACM Council with a petition containing at least 10 signatures or the signatures of 10\% of the members, whichever is smaller. The decision will then be settled by a Standard Vote of the Members as described in section 5.

\section{Special Interest Groups}

Special Interest Groups, or SIGs for short, exist to provide a way for members to start groups within the ACM in order to further the goals of the ACM and help it better fulfill it's purpose as laid out in section 2. They allow ordinary members to gather together and educate each other on topics relevant to the members of the ACM.\\
\\
Any member of the ACM in good standing may start a Special Interest Group.\\
\\
If a SIG holds no meetings for two consecutive weeks during the fall or spring semester, a Decision (section 6) may be passed to dissolve the SIG. The SIG may be restarted by the same process as starting a new SIG.\\
\\
SIGs are not expected to hold meetings during any of the summer semesters nor any day that the University is not holding classes. 

\subsection{Starting a Special Interest Group}

Any member of ACM in good standing may start a SIG.\\
\\
In order to start a SIG one must first submit a document to the ACM Council detailing the name of the SIG, the purpose of the SIG, and laying out a system for deciding who will be the SIG representatives on the ACM Council as well as how to decide what the SIG's vote is for matters on the Council. After submitting this paperwork a SIG becomes an Unofficial SIG with the submitter(s) of the paperwork as the SIG Leader by default.\\
\\
An Unofficial SIG becomes an Official SIG immediately after it holds at least one meeting with attendees present who are not the default SIG Leader(s) and goes through the process laid out in the SIG's defining paperwork to decide on its Leader(s).

\section{Amendments}

Amendments to this document may be proposed by any member who submits a petition to the ACM Council containing at least 10 signatures or the signatures of at least 10\% of the members of the ACM, whichever is smaller.\\
\\
After the petition is submitted this will trigger a Standard Vote of the Members as described in section 5 with the following addendums:
\begin{itemize}
	\item Voting will take place over the course of 10 weekdays.
	\item For the vote to count at least 25\% of current members must cast a ballot.
\end{itemize}

\section{The Judicial Council}

The power to decide if Decisions (section 6) are valid under this document lies with a council made up of the Faculty Adviser, Past President, and current ACM President. This council will be called the Judicial Council.\\
\\
Any member in the organization may contact any of the members of the Judicial Council to call into question any Decision (section 6) made by the ACM Council.\\
\\
By a unanimous vote of the Judicial Council, they may nullify any Decision (section 6) made on the grounds that it is unconstitutional.\\
\\
Judicial Council meetings are open to all members of the ACM.\\
All those present at the meeting must be given equal opportunity to give input on each agenda item.

\section{Dissolution}

In the event of the dissolution of the ACM, all of the assets of the ACM will be transferred to the National ACM.

\end{document}
